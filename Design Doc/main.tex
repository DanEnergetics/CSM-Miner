\documentclass[12pt]{extarticle}
\usepackage[utf8]{inputenc}
%\usepackage{cite}
\usepackage[autolang=other,backend=biber,dateabbrev=false,sorting=none]{biblatex}
\usepackage{graphicx}
\usepackage{subcaption}
\usepackage{amsmath}
\usepackage{pdfpages}
\usepackage{listings}
\usepackage{float}
\usepackage[format=plain,
            labelfont={bf},
            textfont=it]{caption}
\lstset{
basicstyle=\fontsize{11}{13}\selectfont\ttfamily,
frame=single
}
\usepackage{hyperref}

\addbibresource{ra.bib}

\title{Software Design Descriptoon: CSM Miner}
\author{
Odysseas Karanikas\\
\texttt{odysseas.karanikas@rwth-aachen.de}
\and
Mann, Daniel\\
\texttt{daniel.mann@rwth-aachen.de}
\and
Daniel Rein\\
\texttt{drein99@outlook.de}
}

\date{May 2019}

\begin{document}


\maketitle

\section{Introduction}
\subsection{Purpose}
This document contains the complete design description of the Web Application for CSM Miner. \\
The architectural features of the system consisting of the front-end and the back-end will be explained in detail. \\
The primary audiences of this document are the software developers. 

\subsection{Scope}
As already mentioned, the system consists of two major subsystems communicating with each other. The front-end will be a Web Application where the user can upload an event log, generate a visual representation of this log and can then edit the log and generate different views. \\
The back-end will receive the event log from the Web Application, returning the generated views which are then shown in the Web Application.

\subsection{Overview of the Document}

\section{Architectural Design}

\subsection{Unit Design: Web Application}

\begin{figure}[H]
    \centering
    \begin{subfigure}[b]{0.85\textwidth}
        \includegraphics[width=\textwidth]{img3.jpeg}
        \label{fig:arc_1}
    \end{subfigure}
\end{figure}

\subsubsection{windowHandler}
Description: This class will be able to show and hide HTML Objects. \\ \\
Operations: \\
    void: hide(Element e) \\
    Arguments: An HTML Object \\
	Returns: Nothing returned. \\
	Description: This method will hide the HTML Element e. \\ \\
	void: show(Element e) \\
    Arguments: An HTML Object \\
	Returns: Nothing returned. \\
	Description: This method will show the HTML Element e. \\

\subsubsection{Graphics}
Description: This class will render the graphs. \\ \\
Operations: \\
    Image: render(String json) \\
    Arguments: A json file \\
	Returns: The rendered image of the graph defined in the json file. \\
	Description: This method will geneare a graph according to the json file. \\

\subsubsection{Forms}
Description: This class will handle the forms used. \\ \\
Operations: \\
    void: make() \\
    Arguments: None. \\
	Returns: Nothing returned. \\
	Description: This method will generate a form. \\ \\
	void: edit() \\
    Arguments: None. \\
	Returns: Nothing returned. \\
	Description: This method will edit the appearance of a form. \\ \\
	String: getResult() \\
    Arguments: None. \\
	Returns: A String containing the content of a form. \\
	Description: This method will return the content of a form. \\

\subsubsection{WebClient}
Description: This class will receive the data from the back-end. \\ \\
Operations: \\
    String: get(String url) \\
    Arguments: A String containing the url. \\
	Returns: Returns a json file. \\
	Description: This method will return a json file form the url as a String. \\ \\
	int: getPercentage() \\
    Arguments: None. \\
	Returns: A percentage of the download status. \\
	Description: This method will calculate the current download status and return the percentage. \\

\subsubsection{Graph}
Description: This class will represent a graph. \\ \\
Operations: \\
    jsonObj: parse(String json) \\
    Arguments: A json file. \\
	Returns: A json object. \\
	Description: This method will generate the graph according to the json file and will return a json Object. \\ \\
	String: elementInfo(clickresult) \\
    Arguments: The object the user clicked on. \\
	Returns: The information the user demands. \\
	Description: This method will return the information associated to the object the user clicked on. \\ \\
	void: setLabels() \\
    Arguments: None. \\
	Returns: Nothing returned. \\
	Description: This method will make a webRequest to edit a label in a graph. \\
	
	
\subsection{Unit Design: URLs}

\begin{figure}[H]
    \centering
    \begin{subfigure}[b]{0.35\textwidth}
        \includegraphics[width=\textwidth]{img2.jpeg}
        \label{fig:arc_1}
    \end{subfigure}
\end{figure}

\subsubsection{urls}
Description: This class will generate responses to received HttpRequests. \\ \\
Attributes:
    String BASE\textunderscore DIR \\
    Description: The String will contain the server directory.
Operations: \\
	String: get\textunderscore projects() \\
    Arguments: None. \\
	Returns: Returns a merged list as a String. \\
	Description: This method will merge all project json files and return it as a String. \\ \\
	String: createProject(String name) \\
    Arguments: The name of the project as a String. \\
	Returns: XXX. \\
	Description: This method will create a new project with a custom name. \\ \\
	HttpResponse: get(request) \\
    Arguments: A request. \\
	Returns: Returns the index.html. \\
	Description: This method will handle all the requests. \\ \\
	HttpResponse: iFrame(request, String fileName) \\
    Arguments: The request and the name of the file. \\
	Returns: Returns a html file. \\
	Description: This method will return a html file used for iframes. \\ \\
	HttpResponse: getImg(request, String fileName) \\
    Arguments: The request and the name of the file. \\
	Returns: Returns a png file. \\
	Description: This method will return a png image from the image folder. \\ \\
	HttpResponse: getJs(request, String fileName) \\
    Arguments: The request and the name of the file. \\
	Returns: Returns a js file. \\
	Description: This method will return a js file used for external resources. \\ \\
	HttpResponse: request(request, String fileName) \\
    Arguments: The request and the name of the file. \\
	Returns: Returns a json file. \\
	Description: This method will generate a json file for a request (e. g. generate Graph, labeling, etc.). \\

\section{Data Structure Design}

\subsection{Back-End}

\begin{tabular}{ | l | l | l |}
    \hline
    Field & Type & Description \\ \hline
    test & test & test \\
    \hline
\end{tabular}

\section{User Interface Design}


\section{Use Case Relations}

\begin{figure}[H]
    \centering
    \begin{subfigure}[b]{0.85\textwidth}
        \includegraphics[width=\textwidth]{img1.jpeg}
        \label{fig:arc_1}
    \end{subfigure}
\end{figure}

\end{document}