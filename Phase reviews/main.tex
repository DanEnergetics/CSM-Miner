\documentclass[12pt]{extarticle}
\usepackage[utf8]{inputenc}
%\usepackage{cite}
\usepackage[autolang=other,backend=biber,dateabbrev=false,sorting=none]{biblatex}
\usepackage{graphicx}
\usepackage{pdfpages}

\addbibresource{pi.bib}

\title{CSM: Project Initiation}
\author{
Odysseas Karanikas\\
\texttt{odysseas.karanikas@rwth-aachen.de}
\and
Mann, Daniel\\
\texttt{daniel.mann@rwth-aachen.de}
\and
Daniel Rein\\
\texttt{drein99@outlook.de}
}

\date{April 2019}

\begin{document}

\maketitle

\section{Phase review}
\subsection{First thoughts on the project}
After our group got assigned with the project we had a very straight forward attitude. The first thing we did after the initial presentation was quickly decide what material we would need for the project and we speculated how the finished product could be build.
\subsection{Phase execution}
 We got ourselves in touch with the provided lecture and met to plan our next step for the phase. The first thing we did is share our skills with each other to discuss the best possible assignments for each of us. Next we planned on how to execute the first phase. After we had found a usable model each of us did their task for the current phase. The results of our planning are viewable in the provided .pdf and .html files.
\subsection{Conclusion}
Our original ideas on how to solve the different parts of the project were not that far from our final decision on how to build the product. This was due to the initial presentation, because even if we had no clue to the inner workings of the algorithms, the live view of the available solution via PROM and the explanations on what we were seeing set the parameters of the project very tightly. 
Our main conclusion of this first phase was to have weekly project meetings, to discuss problems and next steps in person, because the value of being able to quickly share an idea via a drawing of a design element or an graph is not to be underestimated.

\printbibliography

\end{document}