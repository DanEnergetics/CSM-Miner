\documentclass[12pt]{extarticle}
\usepackage[utf8]{inputenc}
%\usepackage{cite}
\usepackage[autolang=other,backend=biber,dateabbrev=false]{biblatex}

\addbibresource{pi.bib}

\title{CSM: Project Initiation}
\author{
Odysseas Karanikas\\
\texttt{odysseas.karanikas@rwth-aachen.de}
\and
Mann, Daniel\\
\texttt{daniel.mann@rwth-aachen.de}
\and
Daniel Rein\\
\texttt{drein99@outlook.de}
}

\date{April 2019}

\begin{document}

\maketitle

\section{Introduction}

This document aims to lay out starting information for the development of an interactive Web-App that Mines Composite State Machines for processes with several perspective.\\
For this, we will develop a Business Case, undertake a Feasibility Study, establish a project charter, appoint the project team, mark down the office set-up and give a review of the initiation phase in the end.

\section{Business Case}

Many process discovery tools mine models that are focused on activities and their ordering in a process. If, however, the process states are explicit in the log data, it is more intuitive to extract models that depict states and state changes \cite{csm-intro}. This project is dedicated to the case where input is of the mentioned kind, that is, it holds explicit state information.\\
In complex systems with a plethora of states and possible state transitions a mined state graph can turn out an intractable net of causations and correlations. Our project aims at making the analysis of such systems easier and more meaningful by taking into account the different components of the system. In such systems, each state is a composition of several different states with regard to each component of the system. An example mentioned in \cite{csm-intro} is the human body. Here the set of states and state transitions is evidently manifold. To make the analysis of processes in the human organism more understandable, one naturally considers different perspectives, that is, one notices that the state of the human individual is comprised of its metabolic state, e.g. hungry, sated or eating, its state of consciousness, e.g. asleep or awake, and so on.\\
Software that can let the user choose different perspectives on processes of the same system by projecting the total state graph on its components will help the user detect processes in her system.\\

The existing software \cite{prom} with this functionality has various use cases ranging from the reverse engineering of processes in the Cyber-Physical Production Systems\cite{cpps} to the analysis of patient flow in healthcare units \cite{patient}. However, the current software is rather slow and bloated. We suggest a reactive and easy-to-use, interactive web interface based on python.

\section{Feasibility Study}

We deem the project feasible from both the theoretical and technical point of view for the following reasons.

\subsection{Theoretical Aspects}

The theory behind composite state machines has already been laid out by van Eck et al. \cite{csm-intro}. A strong argument for the theoretical feasibility is that such a software system has already been implemented by the same team \cite{prom}.

\subsection{Technical Aspects}

The programming language chosen for this project is Python. It has proven to be a well-suited language in the enterprise environment \cite{danjou}. Frameworks like Flask or Django make the easy development of web applications with python possible \cite{framework}. The PM4Py framework provides an extensive Python toolset for process mining applications and will help us build working software without reinventing the wheel.

\subsection{Risks and potential issues}

Every such undertaking brings about a certain set of generic risks purely resulting from the peculiarities of the people involved and their mutual dynamic. Theses risks are of soft sort. They include most basically miscommunication between the members of the team which can express itself in unfairly distributed work and hence a possible decrease of productivity of some members; it can lead to a reduction of harmony up to incompatibility between the components developed or, in the more trivial case, to the failure to abide by the given deadlines. Frequent meetings and honest conversations pose a possible remedy to this problem. Miscommunication is even more prone to happen between the project team and the advisor. This will lead to failure of the advisors requirements; it can be remedied by an extensive requirement analysis and frequent meetings with the advisor from early on in the project.\\
The risks that this particular software project brings with itself lie in its problem and objective. The topic of Process Discovery is vast and new to most of the team members. The projection of a state graph onto its composite states might pose difficult to conceptualize and put into algorithms. Also the process mining framework PM4Py and potential web frameworks Flask and Django are new. Frequent consultations of the advisor and a study of appropriate literature and documentation are uncircumventable for the success of the endeavour.

\section{Project Team}

\subsection{Member: Daniel Mann}

Mann takes the Bachelor Study Computer Science at the RWTH. He has not attended any courses on Process Discovery in particular, however, his focus of study is AI and Machine Learning which gives him a general understanding of data scientific topics. His programming language of choice is Python but web frameworks and PM4Py are new to him.

\printbibliography

\end{document}
